\documentclass[12pt]{amsart}
\usepackage{amsmath}
\usepackage{amsthm}
\usepackage{amsfonts}
\usepackage{amssymb}
\usepackage{ebproof}
\usepackage[margin=1in]{geometry}
\usepackage{hyperref}
\hypersetup{
    colorlinks=true,
    linkcolor=blue
}

\theoremstyle{definition}
\newtheorem{theorem}{Theorem}[section]
\newtheorem{lemma}[theorem]{Lemma}
\newtheorem{definition}[theorem]{Definition}
\newtheorem{corollary}[theorem]{Corollary}
\newtheorem{proposition}[theorem]{Proposition}
\newtheorem{conjecture}[theorem]{Conjecture}
\newtheorem{remark}[theorem]{Remark}
\newtheorem{example}[theorem]{Example}
\newtheorem{problem}[theorem]{Problem}
\newtheorem{notation}[theorem]{Notation}
\newtheorem{question}[theorem]{Question}
\newtheorem{caution}[theorem]{Caution}

\begin{document}

\title{Homework 1}

\maketitle

% problems taken from logic and proof by avigad et al
\begin{enumerate}

\item Here is another (gruesome) logic puzzle by George J. Summers, 
called “Murder in the Family.”

Murder occurred one evening in the home of a father and mother
and their son and daughter. One member of the family murdered
another member, the third member witnessed the crime, and the
fourth member was an accessory after the fact.

\begin{enumerate}
\item The accessory and the witness were of opposite gender.
\item The oldest member and the witness were of opposite gender.
\item The youngest member and the victim were of opposite gender.
\item The accessory was older than the victim.
\item The father was the oldest member.
\item The murderer was not the youngest member.
\end{enumerate}
Which of the four—father, mother, son, or daughter—was the murderer?

Solve this puzzle, and write a \textit{clear} argument using a 
deduction table to demonstrate your reasoning.

\item Using the mnemonic $F$ (Father), $M$ (Mother), $D$ (Daughter), 
$S$ (Son), $K$ (Murderer), $V$ (Victim), $W$ (Witness), $A$ 
(Accessory), $O$ (Oldest), $Y$ (Youngest), we can define propositional 
variables like $FK$ (Father is the murderer), $DV$ (Daughter is 
the victim), etc.
Notice that only the son or daughter can be the youngest, and only the mother
or father can be the oldest.

With these conventions, the first clue can be represented as
\begin{displaymath}
FA \lor SA) \to (MW \lor DW)) \land ((MA \lor DA) \to (FW \lor SW))
\end{displaymath}
in other words, if the father or son was the accessory, then the mother or
daughter was the witness, and vice-versa. Represent the other five clues in a
similar manner.

Representing the fourth clue is tricky. Try to write down a formula that
describes all the possibilities that are not ruled out by the information.

\item Write down a natural deduction proof for $A \land (B \land C) \vdash 
	(A \land B) \land C$. 

\item Write down a natural deduction proof for $Q \vdash (Q \to R) \to R$.

\item Write down a natural deduction proof for $A \lor B \to B \lor A$.

\end{enumerate}

\end{document}
